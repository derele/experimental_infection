




\section{Methods}

\subsection{Experimental infection of eels}

\textit{An. anguilla} were obtained from the Albe-Fishfarm in
Haren-R\"utenbrock, Germany. \textit{An. japonica} were caught at the
glass eel stage in the estuary of Kao-ping River, Taiwan by
professional fishermen and kept at a water temperature of
26$^{\circ}$C until they reached a size of $>$ 35 cm.

The absence of infections with \textit{A. crassus} in both eel species
was confirmed by dissection of 10 individuals of each species.

After an acclimatisation period of 4 weeks (\textit{An. anguilla}) or
when they reached a size of $>$ 35cm (\textit{An. japonica}) eels were
infected using a stomach tube as described in
\cite{boon1990effect}. During the infection period water temperature
was held constant at 20$^{\circ}$C. Eels were kept in 160-litre tanks
in groups of 5-10 individuals and continuously provided with fresh,
oxygenated water and once every two days with commercial fish pellets
(Dan-Ex 2848, Dana Feed A/S Ltd, Horsens, Denmark) \textit{at
  libitum}.

L2 larvae used for the infection were collected from the swimbladders
of wild yellow and silver eels from the River Rhine near Karlsruhe and
from Lake M\"uggelsee near Berlin in Germany. Taiwanese larvae were
obtained from eels from an aquaculture adjacent to Kao Ping River in
south Taiwan and from a second aquaculture in Yunlin county,
approximately 150 km further north on the west coast of Taiwan. They
were stored at 4$^{\circ}$C for no longer than 2 weeks before copepods
were infected. Mixed species samples of uninfected copepods were
collected from a small pond near Karlsruhe, known to be free of eels
(and \textit{Anguillicola}). They were infected individually in wells
of microtiter plates at an intensity of roughly 10 L2-larvae per
copepod. One week after infection they were placed in bigger
tanks. Twice a week yeast was provided as food and at 21 dpi infective
L3 were harvested with using a tissue potter as described by
\cite{haenen_improved_1994}. 50 L3 for infection of individual eel
were suspended in 100 $\mu$l RPMI-1640 medium (Quiagen, Hilden,
Germany) and eels were infected as described above.

55-57 days post infection (dpi) eels were euthanized and dissected.
The swimbladder was opened and after determination of the sex of adult
worms under a binocular microscope (Semi 2000, Zeiss, Germany), adult
\textit{A. crassus} were immediately immersed in RNAlater (Quiagen,
Hilden, Germany).

\subsection{RNA extraction and preparation of sequencing libraries}

RNA was extracted from 12 individual female worms and for 12 pools of
male worms using the RNeasy kit (Quiagen, Hilden, Germany) (see
table).

The paired-end TruSeqTM RNA sample preparation kit (Illumina) was
followed to build sequencing libraries with insert sizes of roughly
270 bp for paired-end sequencing from cDNA libraries: briefly, poly-T
oligo attached magnetic beads were used for purification of mRNA and
to simultaneously fragment the RNA. The RNA was then primed with
random hexamer primers for cDNA synthesis and reverse transcribed into
first strand cDNA using reverse transcriptase. The cDNA was cleaned
from the second strand reaction, overhangs were repaired to form blunt
ends, a single ``A''-nucleotide was added at the 3' end and paired end
sequencing adapters ware ligated with a complementary
``T''-overhang. In this step multiple differently indexed paired-end
adapters were used to enable multiplexing of the 24 different
sequencing libraries in 3 pools of 8 samples each. These three pools
all contained one random replicate each for each treatment combination
ensuring complete statistical independence of replicates. Molecules
having adapter sequences were enriched in the mix using PCR and the
libraries were controled for quality and quantity on the BioAnalyzer
(Agilent). Clusters were generated by bridge amplification. The
resulting clusters were sequenced on the Genome Analyzer IIX in
combination with the paired-end module. The first read was sequenced
using using the first primer Rd1 SP. The original template strand was
then used to regenerate the complementary strand, the original strand
was removed and complementary strand acted as a template for the
second read, sequenced primed by the second sequencing primer Rd2 SP.

\subsection{Quantitative PCR validation}

!!!

\subsection{Mapping and normalisation of read counts}

All sequencing reads were mapped to the fullest 454 assembly described
in !!!454paper!! and we excluded TUGs inferred as host or cobiont
origin as filter, using BWA \cite{pmid20080505} (version 0.5.9-r16;
BWA aln and BWA sampe with default options) and processed with
samtools \cite{journals/bioinformatics/LiHWFRHMAD09} (version 0.1.18;
samtools view -uS -q 1) to only allow uniquely mapping reads. All
reads mapping to host and cobiont off-target data were removed during
downstream evaluation.

Counts were summed for technical replicates and counts to lowCA
derived contigs (see !!!454paper!!!) were disregarded as well as
spurious read counts to contigs with less than 32 mapping reads in
total (see however \ref{collapse} for how these counts were used in
further tests of reference fragmentation).

The remaining counts were normalised using DESeq (version 1.6.1) (i.e.
the normalisation factor was estimated by the median of scaled counts,
similar to the weighted trimmed mean of the log expression ratios used
later in edgeR). All tables summarising read-counts are based on these
normalised counts. We obtained ``variance stabilised data'' in an
expression matrix for each gene and library using the ``blind'' option
in a calculation not informed (and biased by) the model-design. These
data were used in all gene-centring heatmap and multivariate
visualisations. Additionally this matrix was transposed to get
sample-to-sample distances.

\subsection{Statistical analysis with generalised linear models (GLMs)}

The R-package edgeR (version 2.4.1) \cite{pmid19910308} was used to
build negative binomial generalised linear models, as these
specialised GLMs outperformed GLMs in DESeq in speed and reliability
of convergence. Modeled were based on a negative binomial distribution
and the dispersion parameter for each transcript was approximated with
a trend depending on the overall level of expression. In the maximal
fitted model expression was regressed on worm-sex, host-species and
parasite population, including all their interactions. The full model
thus contained terms $ S_i + H_j + P_k + (SH)_{ij} + (SP)_{ik} +
(HP)_{jk} + (SHP)_{ijk} + \varepsilon$, where $\varepsilon$ is the
residual variance, $S_i$ is the effect of the ith sex (male or
female), $H_j$ is the effect of the ith host species
(\textit{An. anguilla} or \textit{An. japonica}), $P_k$ is the effect
of the kth population (European or Asian), $(SH)_{ij}$ is the
sex-by-species interaction and similarly for the other interactions.

The hierarchical nature of generalised linear models was respected
considering (removing) all interaction effects of a main-term
(e.g. $(SP)_{ik}$, $(SH)_{ij}$ and $(SHP)_{ijk}$) when analysing
models for the significance of that term (e.g. $S_i$). Resulting
p-values were corrected for multiple testing using the method of
Benjamini and Hochberg \cite{benjamini1995controlling} and
differential expression was inferred at a false discovery rate (FDR)
of 5\% (adjusted p-value of 0.05).

\subsection{Count-collapsing for orthologs from two model-species}
\label{collapse}

In order to test the influence of deficiencies (i.e. fragmentation) of
the assembly on mapping we summed read counts over orthologous
sequence in \textit{C. elegans} and \textit{B. malayi}.  Differential
expression for these orthologous-counts was analysed the same way as
for contigs. Contigs were filtered based on inference from orthologous
counts merging the two orthologous evaluations and the contig
evaluation. Differential expression was accepted at a FDR of 5\% for
the contig evaluation and 10\% for both of the two orthologous
evaluations.

\subsection{Multivariate confirmation of linear models}

I used the R package vegan (version 2.0-2) to perform constrained
redundancy analysis on contigs identified as significant in GLMs
before. For each set of contigs (different for sex, eel-host or
worm-population) the appropriate constrained component was used. The
proportion of the variance explained by the constrained component was
recorded and the constrained component was tested for significance
using a permutation test implemented in vegan.

\subsection{GO-term enrichment analysis}

Prior to analysis of GO-term over-representation (based on dn/ds or
expression values) we used the R-package annotationDbi
\cite{AnnotationDbi} to obtain a full list of associations (also with
higher-level terms) from annot8r-annotations. We then used the
R-package topGO \cite{topGO} to traverse the annotation-graph and
analyse each node in the annotation for over-representation of the
associated term in the focal gene-set compared to an appropriate
universal gene-set (all contigs with dn/ds values or all contigs
analysed for gene-expression) with the ``classic'' method and Fisher's
exact test.

\subsection{Clustering analysis}

The R package HeatmapPlus was used on variance stabilised expression
values to visualise hierarchical clusters similar to the method of
\cite{pmid9843981}. The results were displayed along with annotations
stored in a Bioconductor eSet-class object.

\section{General coding methods}

The bulk of analysis (unless otherwise cited) presented in this paper
was carried out in R \cite{R_project} using custom scripts. We used a
method provided in the R packages Sweave
\cite{lmucs-papers:Leisch:2002} and Weaver \cite{weaver} for
``reproducible research'' combining R and \LaTeX code in a single
file. All intermediate data files needed to compile the present paper
are provided at For general visualisation we used the R packages
ggplot2 \cite{ggplot-book} and VennDiagram \cite{pmid21269502}.


\section{Results}


\section{Discussion}


% It is one of the dangers of genomic data to forget the fundamental
% lesson from the debate initiated by Gould and Lewontin in 1979
% \cite{gould_spandrels_1979}. Briefly, while functional changes are
% often caused by selection, differences in function do not necessarily
% demonstrate the past or present action of selection. There is no way
% to infer the action of selection based on functional considerations,
% and even if selection can be inferred otherwise, it is not necessarily
% a particular observed variable trait that selection acted on
% \cite{pmid19744124}.

\subsection{Recovery and adaptation}
\label{sec:recovery}


With some reservations discussed below observation of higher recovery
of adult worms from sympatric \textit{A. crassus}-\textit{Anguilla}
spp. host-parasite combinations imply local adaptation of different
worm populations to host species: Roughly one third of the applied
European worms were recovered form their sympatric host
\textit{An. anguilla} but only little over 10\% for European worms in
\textit{An. japonica}. This pattern of recovery was precisely inverted
for the Taiwanese population of \textit{A. crassus}, for which
recovery was thus roughly 30\% in the sympatric \textit{An. japonica}
and only 10\% in \textit{An. anguilla}. 

Data for the European eel are in agreement with
\cite{knopf_differences_2004} and !!!Weclawski!!!, who did however not
find lower recovery of the European populatio in the Japanese eel.

An ideally suited phenotype to infer local adaptation in a common
garden experiment would be one with or resulting from direct fitness
consequences, a so called fitness component. Fitness is defined as the
differential contribution to the next generation, therefore such a
fitness component would ideally be a measurement on a single
individual, and individual life time reproductive success would be an
ideal measurement. However, techniques to measure such individual life
time reproductive success have not been established in
\textit{A. crassus}.

The recovery of certain developmental stages of worms is only a proxy,
interpretable as a fitness component. It is a composite measurement of
the speed of development from previous lifecycle stages (or speed of
migration towards the swimbladder) and of survival. While survival is
surely an important component of fitness, it is not completely clear
whether fast development and/or migration to the swimbladder are. It
is possible that under certain conditions slower development could
lead to higher fitness, if it would, for example allow development
without attracting the attention of the immune system.

\subsection{Divergence not modification of expression}
\label{sec:sample-twelve}



I decided on a study design using pools of individuals for one sex
(males) and single individuals for the other. A study on
\textit{Fundulus heteroclitus} revealed that approximately 18\% of the
transcripts are differentially expressed between individual fish from
the same population, grown under controlled environmental conditions
\cite{pmid12219088}. And it thus not surprising that between
individual variation in female samples was leading to higher variance
of these female samples compared to pooled male samples in my study.

This interindividual variation in gene expression under a particular
environmental condition is generally agreed to be closely linked to a
genetic basis \cite{pmid15498452}. For example in a cross between two
parental strains of yeast the genetic component of variation was
estimated from haploid segregants to be 84\% \cite{pmid11923494}. The
genetic component was found to be the main factor determining
expression level variability between two strains, sexes and ages of
\textit{Drosophila melanogaster} for 267 (7\%) from 3,931 genes and at
least 25\% of the transcriptome were estimated to be affected mainly
by genotypic factors in any of the groups
\cite{pmid11726925}. Variation in the regulation of gene expression is
thought to constitute a major source of evolutionary novelty
\cite{pmid11341673}.

A second study from the line of research on \textit{Fundulus
  heteroclitus} \cite{pmid16567645} used genetic relatedness as
inferred from phylogenetics to separate variation in gene expression
in a common experimental environment into a neutral component and a
selected component, this way removing variation most likely accounted
for by the shared neutral evolutionary history. My case of
\textit{A. crassus} is potentially simpler: the investigated European
populations are direct descendants and thus a subset of a Taiwanese
source population. In fact I studied two European and two Taiwanese
populations as a few hundred kilometers between the geographical
origins of the two different locations in Germany and Taiwan probably
constitute a barrier to gene flow in a parasite with an aquatic
intermediate host. However, I treated worms from both European and
Taiwanese populations as replicates (and use the terminology of one
European and one Asian population throughout the text) with the
rationale of increasing variance for random genetic differences and
raising the bar for potentially adaptive differences to be detected.


Gene expression values constitute nothing more than a molecular
phenotype. This phenotype is not necessarily a fitness component.


Given the sampling of only twelve Taiwanese worms the question could
be raised, whether these constitute a representative sample of the
true source population, of which a subpopulation was funding European
populations. A microsatellite study indicated gene flow even between
populations of \textit{A. crassus} separated by thousands of
kilometers in Asia (Japan and Taiwan)
\cite{wielgoss_population_2008}. Given the high interconnectivity of
Taiwanese water systems used for aquaculture both by man build
structural links and anthropogenic exchange of fish, a sampling from
two Taiwanese populations similarly neutrally diverged from the true
European funding population seems very unlikely. The worms sampled
from Taiwan can thus be regarded a sample of the metapopulation
appropriate for finding differences in relation to the source of the
introduction.

Of no surprise was the abundance of differential expression between
male and female worms in roughly one third of the genes. A large
number of genes are known to be sex specific, regulating ovulation and
spermatogenesis throughout the metazoa and especially in nematodes
\cite{pmid15371532}. On top of these sex specific genes there are
large numbers of genes differently expressed due to differences in
metabolism between males and females. Estimates for
\textit{Drosophila} based on similar sample sizes to those used in my
study range between one and two thirds of the transcriptome showing
sex biased expression \cite{pmid11726925}. In the liver transcriptome
of \textit{Mus musculus}, even 70\% of transcripts have been shown to
differ between sexes \cite{pmid16825664} (note however that this study
used 169 female and 165 male mice to guarantee the finding of even the
most subtle differences). Given the scale of these differences in
other species my estimate of roughly one third of the transcripts in
\textit{A. crassus} showing differential expression according to the
sex of the worms implies conservative thresholds used in the
statistical analysis and moderate power for detection of differences.

Nearly the same proportion (roughly 30\%) of contigs was confirmed
through summation and analysis of contigs for orthologs in
\textit{B. malayi} and \textit{C. elegans}. Development of this
orthologous confirmation method was necessitated by the possibly
fragmented and chimeric transcriptome assembly. This introduces
stringent conditions for the detection of significance, as p-value
correction for multiple testing is employed during each analysis (once
for raw counts and twice for orthologous counts). Although the
underlying tests are not independent, the false discovery rate of 5\%
for raw contigs can be expected to be immensely lowered by applying a
FDR of 10\% twice.

In addition biological implications could produce false negatives in
such an evaluation: All genes duplicated in \textit{A. crassus} (a)
and following antithetic expression patterns will be evaluated
negatively, as will duplicated genes in any of the model species (b)
following such a pattern. However, there is no other choice then
applying these stringent conditions to screen for artefacts producing
the same patterns based on mapping to fragmented (a) or chimeric (b)
reference contigs. I think that an evaluation based on this
scrutinised confidence in an assembly previously computed from 454
data is even more appropriate then an analysis solely based on counts
collapsed for orthologs excluding only possible fragmentation
artefacts (as used e.g. in \cite{pmid22084086}).

In general, my statistical analysis aimed to minimise false positives
(type I error) at expenses of possible false negatives (type II error)
and is thus not fully suited to address the proportions of
differentially regulated genes.

Nevertheless it is surprising that less than 1\% of transcripts were
detected differentially expressed between worms in different host
species and less then 0.3\% were confirmed with the orthologous
summation method. This was an unexpected finding, as the differences
in the immune response of the host species have a big influence on
other phenotypes of worms \cite{knopf_swimbladder_2006}. In addition
to the low number of genes, multifactorial analysis revealed that
below 10\% of the variance could be explained by host species effect,
even in significantly differential regulated genes for this factor.

Although these differences between worms in different host species
were the most marginal of any of the factors, it is possible to
connect some (at least two) of the genes to a prominent physiological
difference: the digestion of haemoglobin. Two different aspartic
proteases (both confirmed through orthologs, one of them differing for
all three main effects, the other for an interaction of worm sex and
host species) known to be involved in the first steps of digestion of
haemoglobin from other nematodes \cite{pmid12782060} were
overexpressed in worms in \textit{An. anguilla}. This expression
phenotype could potentially be linked to the often observed phenotype
of bigger size of \textit{A. crassus} in this host
\cite{knopf_swimbladder_2006}, as the main contribution to this
increase in size is the larger volume of host blood taken up by the
parasite. Accordingly the parasite probably digests haemoglobin at a
higher rate.

Close to 1\% of contigs were significantly different in expression
between European and Asian \textit{A. crassus}, making this difference
significant for a higher number of contigs than the host
differences. For this contrast the proportion of orthologous
confirmation was lower than for sex differences but higher than for
host species differences. Additionally multivariate analysis of all
differently expressed transcripts for worm population revealed that
the variance contributed by the population factor was higher than 10\%
for all significant contigs or even 20\% for orthologous confirmed
contigs.

Another important finding was the large overlap in contigs expressed
differentially depending on worm sex and worm population. Such an
overlap is expected if genes expressed differentially according to sex
are evolving faster towards a differential expression according to
other factors. Faster evolution of reproductive (and especially male
specific) traits has been shown in many species at a phenotypic and at
a sequence level \cite{pmid15795858}. In \textit{Drosophila}, male
reproductive proteins have been shown to evolve at elevated levels and
under positive selection \cite{pmid11404480}. Moreover, gene
expression should evolve at a higher rate in sex specific genes.
Indeed the transcriptomes of \textit{Drosophila} species show that
interspecific expression divergence is sex dependent and the action of
sex dependent natural selection during species divergence has been
inferred from this \cite{pmid15034135,pmid19720861}.

Taken together, my findings strongly support a stronger influence of
genetic differences between European and Asian populations of
\textit{A. crassus} than of the modification in the different host
species on gene expression. When additive and interaction effects are
considered, the influence of host species even vanishes almost
completely in favour of a combination of effects combining parasite
population and sex of the worms.

\subsection{Functions of genes with genetically fixed expression
  differences}
\label{sec:function-genes-with}

From a functional perspective, genes identified to differ between
populations can be categorised as important in general metabolic
processes instead of specific host parasite interactions. This
constitutes a negative evaluation of one of my \textit{a priori}
hypotheses based on finding parasite specific genes, identified as
vaccine candidates in a number of nematodes, within the genes modified
or diverged in my study (\ref{sec:dna-sequ-nemat}). However, more
direct host parasite interactions are expected in tissue dwelling
larval stages (L3 and L4) and in fact most immunomodulators are
expressed predominantly in these stages
\cite{maizels_helminth_2004}. Adults of \textit{A. crassus} could thus
be the wrong lifecycle stages to detect such expression differences,
if they existed.

\subsubsection{Metabolism}
\label{sec:metabolism}

Instead enzymes and enzyme subunits important for aerobic respiration
are especially expressed at lower levels in European
\textit{A. crassus}. In fact, most transcripts significantly differing
between populations were annotated as ``oxidoreductase'' in gene
ontology (GO).
% While at face values of single genes have to be interpreted with
% care such trends in the data are very likely to point to important
% differences.
Downregulation of cytochrome C oxidase subunit 2 (COXII) in the
European population of \textit{A. crassus} was the most persistent
finding. This downregulation was confirmed by the low expression of
the same contig in the European libraries compared to higher
expression in all three libraries from Taiwanese worms in
pyrosequencing. Cytochrome C oxidase subunits 1-3 are are essential
components of respiratory chain complex IV, the cytochrome c
oxidase. They are encoded in the mitochondrial genome and coordinate
catalytic heme and copper cofactors \cite{pmid18023115}.

In fact, not only enrichment analysis highlighted oxidoreductases, but
expression values of COXII clustered with other enzymes related to the
state of energy metabolism: two lecitin:cholesterol acyltransferase
transcripts are putative recently duplicated genes. They showed
slightly divergent protein sequences but hit the same orthologs in
\textit{C. elegans} and \textit{B. malayi}. They also shared very
similar expression profiles. Expression of different cholesterol
acyltransferases has been shown to vary in response to the presence of
heme and anaerobiosis in yeast \cite{pmid11786267}. 3-hydroxyacyl-CoA
dehydrogenase (involved fatty-acid $\beta$-oxidation
\cite{pmid8454629}), malate/L-lactate dehydrogenase (from the
anaerobic glycolytic pathway or the Krebs-cycle
\cite{sturm1969vergleichende}) and aspartyl proteases (involved in the
digestion of host haemoglobin in helminths \cite{pmid12782060})
completed this particular cluster.

These patterns can be interpreted as a biological confirmation of the
at face values for single genes, especially for COXII. In addition the
differential reaction of metabolic genes to different factors (genetic
vs. modification) invites speculation on a causal structure behind
these correlations. The expressions of metabolic enzymes are
interpretable as a change to use a more anaerobic metabolism in the
European population of \textit{A. crassus}. In one possible scenario,
in European worms one of the subunits of core enzymes of the
respiratory chain (probably COXII) would have evolved a genetically
fixed lower level of expression. This model follows the logic that the
most differential expressed gene could be the driver of observed
change. Other enzymes related to aerobic energy metabolism directly or
indirectly via the redox state of cells (e.g. lipid metabolism) and
only partially controlled by feedback mechanisms from oxidative
phosphorylation and the citric acid cycle would show similar patterns
of altered expression in European worms. However, the expression of
these indirectly and also by additional environmental factors
controlled genes would be perturbed when worms are applied back to
their Asian hosts. Also in the two sexes differences in size and
metabolism would be perturbing the pleiotropic effects of the
persistent core change.

Such a scenario also provokes speculation about the adaptive value of
such a change in a core metabolic process: aerobic respiration is a
potential source for oxidative stress providing a steady source of
reactive oxygen species (ROS) as electrons are leaking from the
respiratory chain as superoxide anions. It is well established that
such ROS production is especially harmful to blood feeding parasites,
as free inorganic iron, as well as heme, have the potential to
generate additional ROS \cite{pmid21087517}. Anaerobic metabolism is
thus thought to occur in many haematophagous parasites as a counter
measure against oxidative stress from haemoglobin catabolism
\cite{pmid12163151}. It could thus be hypothesised that the bigger
size and the larger amount of eel blood ingested leading to a higher
rate of haemoglobin digestion provided the selective pressure to
reduce aerobic respiration. Additionally helminths can simply get too
large to maintain oxygen diffusion to mitochondriae in the absence of
a cardiovascular system. As yet proton pumping electron transport
constitutes the most profitable energy providing process, the
mitochondriae of facultatively anaerobic helminths produce a proton
gradient for the use of ATPase with the help of terminal electron
acceptors other than O$_2$ \cite{pmid12417132}. Such an alternate
electron sink is fumarate used in many helminths in a process called
malat dismutation \cite{pmid15275412}.

An interesting implication is that such metabolic differences could
potentially be visible ultrastructurally. Indeed in my own diploma
thesis \cite{heitlinger_vergleichende_2008} I identified two different
kinds of mitochondriae, one with standard christae like morphology,
the other with unusual sacculus like morphology in
\textit{A. crassus}. Additionally I observed less electron dense
inclusions (probably lipid reserves) in bigger worms and more glycogen
granulae. The fact that such lipids are less usable under anaerobic
conditions led me to the hypothesis that bigger worms are using less
aerobic processes. Reanalysing this data and probably obtaining new
data with additional histochemical staining methods could be a way to
put gene expression into a physiological perspective. Furthermore, a
biochemical examination of isolated mitochondriae could highlight
changes in the mitochondrial respiratory chain under \textit{in vitro}
conditions \cite{pmid18314717}. Such direct measurements of COX enzyme
activity (using well established assays \cite{pmid8592440}) would be
desirable to establish even the validity of the first logical step in
these adaptive speculations that underexpression of COXII is leading
to decreased enzyme activity. It would be counterintuitive to expect
higher enzyme activity when COXII mRNA levels are low, but, for
example, in \textit{Schistosoma mansoni} COXI over expression in
praziquantel resistant strains is leading rather to decreased enzyme
activity \cite{pmid9695101}.

The sensitivity to perturbation of mitochondrial genes for respiratory
chain complexes in nematode parasites is underlined by their
upregulation after depletion of Wolbachia from filarial nematodes
\cite{pmid20362581, pmid19806204}. Wolbachia are obligate endosymbiont
bacteria of some clade III nematodes, they are supplying heme to
non-haematophagous parasites in the absence of an intrinsic pathway
for heme synthesis \cite{ghedin_draft_2007} (which is absent also in
free living \textit{C. elegans} \cite{pmid15767563}). While my
sequence analysis suggests the absence of wolbachial symbionts in
\textit{A. crassus}, such studies support a central role of host or
endosymbiont derived heme for respiratory processes and suggest a
propensity for evolutionary change in related processes (in Filaria
even acquisition of an endosymbiont).

Assuming a genetically fixed lower expression of COXII in European
\textit{A. crassus} as a driver for other metabolic differences does
not imply a simple regulation of the expression itself, or a
genetically simple change underlying the changed expression
phenotype. Regulation of the mitochondrially encoded genes has been
extensively integrated into the regulatory network of eukaryotic
cells and is controlled by and interacting with nuclear transcription
factors \cite{pmid8289797}.

Intriguingly overexpression of respiratory chain enzymes was limited
to cytochrome c oxidase transcripts (COXII and to lesser extent also
COXI and COXIII). Mitochondrial transcription produces multiple
polycistronic unmatured transcripts, which are cleaved and modified in
their expression post-transcriptionally. Cleavage occurs at t-RNA
sequences interspersed between protein coding genes and can be
imperfect to leave some transcripts polycistronic in a matured
state. Nevertheless, due to posttranscriptional modification
individual transcripts can be expressed uncoordinated, even when
expressed on the same unmatured polycistronic transcript
\cite{pmid19843606}. The addition of poly-A tails, for example, is
vital for stability of mature transcripts in metazoans. The
mitochondrial genome contains only very little untranscribed sequence,
is polyploid (once homoplasmic, essentially maternally inherited like
haploid) and transmitted completely linked, with very scarce
recombination events \cite{pmid18023115}.

Cis-regulatory change in a control region would thus be very easily
detectable in my transcriptome data. Even if the sequence variation
leading to the observed expression phenotypes would locate to the
untranscribed hypervariable mitochondrial control region (in D-Loop
associated promoters), selection on such a variant would render the
whole mitochondrial genome inadequate for phylogenetic analysis, as a
variant sweeping to fixation would have removed polymorphism from the
complete mitochondrial genome due to the prefect linkage
\cite{pmid19821901}. If a sweep would be presently ongoing, high
levels of heteroplasmy would be found in single individuals
\cite{pmid21226948}. Such a pattern has not been found in populations
of \textit{A. crassus} in Europe when COXI was used as a marker
\cite{wielgoss_population_2008, dl_py} (see also figure
\ref{mCOXI-phylo}) and is also not visible from preliminary analysis
of polymorphism in mitochondrial genes in my RNA-seq data.

Functional constraints are also expected regarding the mechanism by
which the expression of COXII could evolve. Most infective L3 larvae
of parasitic nematodes rely on aerobic respiration
\cite{kennedy2001parasitic}. Dixenous parasites like
\textit{A. crassus} migrate through tissues of definitive hosts, where
oxygen is readily available, after leaving the haeomocoel of the
intermediate host. Enzyme subunits building a functioning aerobic
respiratory chain are thus likely to be expressed at earlier lifecycle
stages of \textit{A. crassus} and elevated anaerobiosis is expected to
be restricted to the adult stages.

These considerations make sole or predominant cis-regulatory change in
mitochondrial DNA unlikely to explain the divergent expression
phenotypes. Still identification of the genetic architecture, for
example sequence variation in a transcription factor, a co-factor or a
protein modifying mitochondrial transcripts, may be possible (to a
limited extent even in the present RNA-seq data).

RNAi screens in \textit{C. elegans} for increased lifespan focus on
genes leading to lower oxygen consumption and altered mitochondrial
morphology and function \cite{pmid12447374}. Such candidate genes will
provide an additional link back to functional considerations once
screening for genomic regions with signature of selection will
highlight candidate loci.

% Numts should normally be avoided by reverse transcription, but
% occasionally Numts are transcribed

% (Blanchard, J.L. and Schmidt, G.W. (1996) Mitochondrial DNA migration
% events in yeast and humans: Integration by a common end-joining
% mechanism and alternative perspectives on nucleotide substitution
% pattern. J. Mol. Evol. 13, 537–548).

% Pseudo-mRNAs shadowy entities that resist classification and analysis
% resemble protein-coding mRNA, but cannot encode full-length proteins
% owing to disruptions of the reading frame.  \cite{pmid16683022}

\subsubsection{Collagens}
\label{sec:collagens}

A second group of genes differentially expressed in populations of
\textit{A. crassus} emerged from both cluster and enrichment
analyses. Two transcripts in this cluster were significant for
interaction effects between host species and parasite-population, they
were annotated as collagens. For both genes this meant an ``adjusted''
(to avoid the suggestive ``adapted'') expression difference leading to
a lower expression in sympatric host species/parasite population
pairs. Cuticle collagens are a large multigene family (Interpro lists
164 entries for ``Nematode cuticle collagen, N-terminal'' for
\href{http://www.ebi.ac.uk/interpro/ISpy?ipr=IPR002486&tax=6239}{\textit{C. elegans}
} and 51 for
\href{http://www.ebi.ac.uk/interpro/ISpy?ipr=IPR002486&tax=6279}{\textit{B. malayi}}),
containing extensive repeat regions: roughly 50\% Gly-X-Y residues,
often Gly-Pro-Hpy. In the genome of \textit{B. malayi} 82 genes
encoding collagen repeats have been found \cite{ghedin_draft_2007}. It
was thus very important to have orthologous confirmation for these two
contigs, as misassembly could have easily lead false positives here.

The two collagens were clustered with a third contig sharing a
collagen annotation (failing to be significant for the interaction
term probably because of low overall expression) and a contig
annotated as ``Matrixin'' (a metallo-proteinase assumed to be involved
in remodelling of the extracellular matrix \cite{mealloprot}) and a
ABC-transporter family protein.

Functional speculations are more difficult for collagen than for the
respiratory chain enzymes. The cuticle constitutes an exoskeleton and
a barrier between the worm and its host environment. Synthesis of most
collagens is believed to occur at negligible levels in adult male
worms and is rather constrained to discrete temporal periods in larval
development, the moults \cite{pmid10637627}. The differential
expression could thus be due to changes in larval development or due
to alternations in the low level, steady renewal of the adult cuticle
and remodelling of the extracellular matrix of hypodermis cells. Some
considerations would favour of the second explanation: in
\textit{C. elegans} genes expressed after reproductive maturity evolve
faster than genes expressed earlier in development
\cite{pmid15371532}. This suggests a model of elevated pleiotropic
effects in genes expressed at earlier stages of development and hence
more conserved expression patterns in larval stages. Independent of
these considerations, both the primary assembly and the constant
remodelling of the cuticle involve complex post-translational
processes hardly accessible at the transcriptomic level: a zipper-like
nucleation/growth mechanism leads to the folding of a triple helix of
and heterotrimers and homotrimers \cite{kennedy2001parasitic}. If and
how differential expression of two particular collagens interferes
with this process requests further research. As for the metabolic
differences, differential expression patterns could be reflected in
morphology. One approach would be to measure thickness and density of
the cuticle of worms from coinoculation experiments.

\section{Outlook}

The presented project on the divergence of gene expression obviously
constitutes work in progress. The observed differences in subunits of
respiratory chain enzymes, especially in COXII, necessitate and permit
confirmation by reverse transcription quantitative PCR (RTqPCR) for
these transcripts. Such evaluations of a single gene (or few genes)
will be possible on many individual specimen of \textit{A. crassus}
from both Europe and Taiwan to further test the significance of the
observed differences. Therefore, in addition to the validation of
expression values for sequenced samples, many of the worms from the
presented coinoculation experiment yielding lower amounts of RNA
inadequate for sequencing will be used to further establish the
divergence in gene expression. Additionally sampling of worms from
their present day sympatric hosts is possible for genes differing only
for populations unconditional on eel host species. Moreover, if
selection in Europe would have acted on standing variation, one would
expect to find worms expressing for example COXII at low levels also
in the Taiwanese source populations, at least in low frequency.

An assembly of the mitochondrial genome of \textit{A. crassus} from
preliminary genome-sequencing data (discussed below) and the
identification of the poly-cistronic unmatured and, if present,
matured transcripts (similar to \cite{pmid19843606}), will further
inform and validate the analysis of the expression of mitochondrial
genes. Additionally, disentangling assembly artefacts complicating
mapping from real nuclear or even mitochondrial \cite{pmid20026478}
pseudogenes of mitochondrial genes will help increasing the power of
expression analysis and furthermore permit the analysis of interaction
of such pseudogenes with the expression of functional genes.

Multiple starting points also exist for further functional examination
of metabolic change, as mentioned throughout the text. However, the
search for ultimate causes for evolutionary change \textit{sensu}
\cite{mayr1961cause} will potentially be even more rewarding.

I will expand the RNA-seq analysis presented here to study
allele-specific expression and the association between gene expression
and sequence variants. This kind of quantitative expression trait
locus (eQTL) analysis is possible as both sequence and expression
information are available from the present RNA-seq data. Both simple
cis-acting variation in promoter or enhancer regions, as well as
trans-acting variation can theoretically be detected
\cite{pmid21838806}. To detect trans-acting variants, however, might
be impossible with the (for population studies) relative low number of
sequenced individuals, as it relies on statistical associations
requiring broad sampling. Yet, cis-acting variation, more readily
detectable as allele-specific variation, is unlikely to explain
variations in mitochondrial gene expressions for the reasons discussed
above.

Therefore, large scale meta-population wide sampling must not be
limited to an evaluation of the divergent gene-expression phenotypes,
but has to further elucidate the population genetic relationships
between Taiwanese and European worms. A future research program will
thus need to employ population-scale sampling of genotype data, densely
spread across the genome. Genotyping of many European
\textit{A. crassus} from different populations and comparison with
many individual genomes from different Asian populations will enable
tests for selection: based on the fact that around selected variants
nucleotide diversity is reduced by hitchhiking of neutral variation in
so called selective sweeps \cite{pmid16251466}, a punctual increase of
population differentiation measured by the fixation index F$_{st}$
\cite{wright1949genetical} in regions linked to selected variants can
be measured. Other well established population genetic measurements
include Tajima's D, a measure based on the allele frequency spectrum
\cite{pmid2513255}. When these methods are applied on a genome wide
scale the neutral null-expectation to separate a loss in variability
based on selection from neutral loss due to demography is given by the
diversity across all regions of the genome. A microsatellite study
\cite{wielgoss_population_2008} as well as my own evaluations (based
on pyrosequencing see \ref{sing-w}) and RNA-seq (data not shown)
indicate only a moderate genetic bottleneck caused by the introduction
of \textit{A. crassus} to Europe and thus the necessary neutral
diversity as a background for these tests will be present.

Furthermore statistical models need to be parameterised by divergence
time to disentangle the influence of demography and selection (i.e. to
estimate the effective population size). Reliable estimates for
divergence time are readily available for the introduction of
\textit{A. crassus} to Europe: 60 to 90 generations. As for such a
short period linkage to putatively selected variants will not be
broken down in large blocks, marker density is of minor concern, but
priority should be given to the breadth (many individuals from many
populations) of sampling.

One methods enabling such population wide genotyping emerging from NGS
technology is the sequencing of restriction-site associated DNA (RAD)
markers. Preparation of RAD libraries involves digestion of genomic
DNA with a restriction enzyme. Individually tagged adaptors can then
be ligated to the fragments and individual samples can be pooled. The
choice of restriction enzyme is important to optimise the number of
restriction sites (depth of sampling the genome) relative to the
number of individual samples being investigated
\cite{pmid18852878}. In the case of \textit{A. crassus} this
optimisation also concerns the minimisation of restriction sites in
host-genome, as present in unavoidable contamination. 

The \textit{de novo} assembly of a reference genome for
\textit{A. crassus} will enable the search for such an optimal
restriction enzyme. Preliminary data has been generated for a female
individual of the Polish population on one lane of the Illumina HiSeq
machine, giving 110 million 100 bases long paired-end reads, in total
over 10 gigabases of sequence data.

A preliminary assembly yielded a mean coverage of below 15-fold, for
the \textit{A. crassus} derived contigs. This coverage is surprisingly
low given the large amount of input-data and I will need to construct
improved assemblies informed by the analysis of this preliminary
assembly. A seemingly trivial but nevertheless important prerequisite
for any high-throughput genomic sequencing project on a parasite was
the confirmation that genomic DNA could be obtained sufficiently clean
from other xenobiont DNA.

\figuremacroW{genome_cov_gc}{GC-content and coverage for a preliminary
  genome assembly}{A preliminary assembly of roughly 10 Gb sequence
  data in over 110 million reads. The analysis of GC-content and
  coverage identifies host-contamination at higher GC, but lower
  coverage. Coverage and GC-content separate two distinct
  data-sources: a lower GC/higher coverage nematode subset and a
  higher GC/lower coverage eel subset (confirmed by
  \texttt{Blast}). For this sequencing library only 10-20\% of the
  reads are lost to eel-host derived off-target data. The preliminary
  assembly was provided by Sujai Kumar from Mark Blaxter's lab.}{0.5}

It has been possible to isolate roughly 1$\mu$g of genomic DNA from a
big individual worm. Only ca. 20\% of the DNA were derived from the
genome of the eel-host (see figure \ref{genome_cov_gc}). As only 300
ng of DNA material (with low amounts of contamination with host-blood)
are needed for RAD-sequencing, this can be achieved in most big
specimen of \textit{A. crassus}.

For both reference genome assembly and annotation and for the future
genome-scans I will continue to collaborate with Mark Blaxter's
laboratory at the University of Edinburgh. This group is actively
developing methods especially for RAD-sequencing and applying them to
questions in evolutionary model-species \cite{pmid21681211}.

Another useful strategy enabled by RAD-sequencing is the construction
of a physical genetic map in families of \textit{A. crassus}
(backcross is impossible). In addition to the population scale
approaches outlined above mapping of gene-expression quantitative
trait loci (eQTL) in mapping crosses between the two divergent
expression-phenotypes constitutes a promising route for the
investigation of genomic variants underlying the divergent
expression-phenotypes. Once transcripts can be anchored on genomic
contigs and linkage groups can be constructed to build a physical map
of the genome, a readout for hybrid F2 individuals could even be
transcriptomic data (RNA-seq) providing both genotype and
expression-phenotype.

A prime example for a research program on the evolution of
ecologically important traits is provided by the Stickleback
\textit{Gasterosteus aculeatus}: QTL-mapping has been performed to
fine-map the loss of lateral plates in freshwater populations
\cite{pmid18852878} and parallel adaptation has been investigated
using population genomics \cite{pmid20195501}. Both approaches used
RAD-sequencing. The sophistication and depth of insight available in
such an evolutionary model species is underlined by research on
adaptive reduction of pelvic structures, an evolutionary trajectory
shown to be favoured by the localisation of the underlying change in
an instable region of the genome \cite{pmid20007865}.

The hope to develop a similar research program based on the present
humble thesis seems presumptuous. Nevertheless, making full use of the
advances in sequencing technology it might be possible to rapidly gain
insight into the genomic organisation underlying contemporary
evolutionary change. The present RNA-seq data will be crucial in
achieving this goal, as it will be used to link expression phenotypes
with genomic sequence. An evolutionary leap in a core metabolic
process seems possible.

The ability to evolve via such a leap could even be an evolutionary
old trait retained in \textit{A. crassus} allowing it to colonise new
hosts. Therefore, comparative genomics relating population genetic
processes in \textit{A. crassus} to putatively adaptive change during
the acquisition of new host by other \textit{Anguillicola} species in
evolutionary time constitutes another route of research. If such a
link between microevolutionary processes in \textit{A. crassus} and
the evolution of \textit{Anguillicola}-species would exist, it would
provide general insight in the evolution of parasitic phenotypes.
