\documentclass[10pt]{article}

% Load packages
\usepackage{cite} % Make references as [1-4], not [1,2,3,4]
\usepackage{url}  % Formatting web addresses  
\usepackage{ifthen}  % Conditional 
\usepackage{multicol}   %Columns
\usepackage[utf8]{inputenc} %unicode support
\usepackage{multirow}
\usepackage{hyperref}
%\usepackage[applemac]{inputenc} %applemac support if unicode package fails
%\usepackage[latin1]{inputenc} %UNIX support if unicode package fails
\urlstyle{rm}

\begin{document}

\section{Methods}

\subsection{Experimental infection of eels}

\textit{An. anguilla} were obtained from the Albe-Fishfarm in
Haren-R\"utenbrock, Germany. \textit{An. japonica} were caught at the
glass eel stage in the estuary of Kao-ping River, Taiwan by
professional fishermen and kept at a water temperature of
26$^{\circ}$C until they reached a size of $>$ 35 cm.

The absence of infections with \textit{A. crassus} in both eel species
was confirmed by dissection of 10 individuals of each species.

After an acclimatisation period of 4 weeks (\textit{An. anguilla}) or
when they reached a size of $>$ 35cm (\textit{An. japonica}) eels were
infected using a stomach tube as described in
\cite{boon1990effect}. During the infection period water temperature
was held constant at 20$^{\circ}$C. Eels were kept in 160-litre tanks
in groups of 5-10 individuals and continuously provided with fresh,
oxygenated water and once every two days with commercial fish pellets
(Dan-Ex 2848, Dana Feed A/S Ltd, Horsens, Denmark) \textit{at
  libitum}.

L2 larvae used for the infection were collected from the swimbladders
of wild yellow and silver eels from the River Rhine near Karlsruhe and
from Lake M\"uggelsee near Berlin in Germany. Taiwanese larvae were
obtained from eels from an aquaculture adjacent to Kao Ping River in
south Taiwan and from a second aquaculture in Yunlin county,
approximately 150 km further north on the west coast of Taiwan. They
were stored at 4$^{\circ}$C for no longer than 2 weeks before copepods
were infected. Mixed species samples of uninfected copepods were
collected from a small pond near Karlsruhe, known to be free of eels
(and \textit{Anguillicola}). They were infected individually in wells
of microtiter plates at an intensity of roughly 10 L2-larvae per
copepod. One week after infection they were placed in bigger
tanks. Twice a week yeast was provided as food and at 21 dpi infective
L3 were harvested with using a tissue potter as described by
\cite{haenen_improved_1994}. 50 L3 for infection of individual eel
were suspended in 100 $\mu$l RPMI-1640 medium (Quiagen, Hilden,
Germany) and eels were infected as described above.

55-57 days post infection (dpi) eels were euthanized and dissected.
The swimbladder was opened and after determination of the sex of adult
worms under a binocular microscope (Semi 2000, Zeiss, Germany), adult
\textit{A. crassus} were immediately immersed in RNAlater (Quiagen,
Hilden, Germany).

\subsection{RNA extraction and preparation of sequencing libraries}

RNA was extracted from 12 individual female worms and for 12 pools of
male worms using the RNeasy kit (Quiagen, Hilden, Germany) (see
table XXX).

3 individual female worms from each experimental group were chosen
randomly to give in total twelve females. Additionally, from three
individual male worms, and from 9 pools of male worms RNA was
extracted (see tabled XXX). Pools consisted of worms from one infected
eel individual each. All replicates were derived from infections of
different eel individuals, with one exception from this form of
statistical independence: from \textit{An. japonica} European male
worms as well as a female worm had to be prepared from the same eel
individuals. It was impossible to extract enough RNA from all but the
biggest male worms especially of the Japanese eel/European worm
combination, leaving no other choice. Because of the small size of
male worms it was generally not possible to randomly choose
individuals. Preparation of sufficient amounts of RNA was only
achieved in pools of the biggest individuals. All male worms were thus
chosen for preparation based on large size, even when pools of worms
were used.

The paired-end TruSeqTM RNA sample preparation kit (Illumina) was
followed to build sequencing libraries with insert sizes of roughly
270 bp for paired-end sequencing from cDNA libraries: briefly, poly-T
oligo attached magnetic beads were used for purification of mRNA and
to simultaneously fragment the RNA. The RNA was then primed with
random hexamer primers for cDNA synthesis and reverse transcribed into
first strand cDNA using reverse transcriptase. The cDNA was cleaned
from the second strand reaction, overhangs were repaired to form blunt
ends, a single ``A''-nucleotide was added at the 3' end and paired end
sequencing adapters ware ligated with a complementary
``T''-overhang. 

In this step multiple differently indexed paired-end adapters were
used to enable multiplexing of the 24 different sequencing libraries
in 3 pools of 8 samples each. These three pools all contained one
random replicate each for each treatment combination ensuring complete
statistical independence of replicates from sequencing-lane effects.

Molecules having adapter sequences were enriched in the mix using PCR
and the libraries were controled for quality and quantity on the
BioAnalyzer (Agilent). Clusters were generated by bridge
amplification. The resulting clusters were sequenced on the Genome
Analyzer IIX in combination with the paired-end module. The first read
was sequenced using using the first primer Rd1 SP. The original
template strand was then used to regenerate the complementary strand,
the original strand was removed and complementary strand acted as a
template for the second read, sequenced primed by the second
sequencing primer Rd2 SP.


\subsection{Quantitative PCR validation}

!!!

\subsection{Mapping and normalisation of read counts}

All sequencing reads were mapped to the fullest 454 assembly described
in !!!454paper!! and we excluded TUGs inferred as host or cobiont
origin as filter, using BWA \cite{pmid20080505} (version 0.5.9-r16;
BWA aln and BWA sampe with default options) and processed with
samtools \cite{journals/bioinformatics/LiHWFRHMAD09} (version 0.1.18;
samtools view -uS -q 1) to only allow uniquely mapping reads. All
reads mapping to host and cobiont off-target data were removed during
downstream evaluation.

Counts were summed for technical replicates and counts to lowCA
derived contigs (see !!!454paper!!!) were disregarded as well as
spurious read counts to contigs with less than 32 mapping reads in
total (see however \ref{collapse} for how these counts were used in
further tests of reference fragmentation).

The remaining counts were normalised using DESeq (version 1.6.1) (i.e.
the normalisation factor was estimated by the median of scaled counts,
similar to the weighted trimmed mean of the log expression ratios used
later in edgeR). All tables summarising read-counts are based on these
normalised counts. We obtained ``variance stabilised data'' in an
expression matrix for each gene and library using the ``blind'' option
in a calculation not informed (and biased by) the model-design. These
data were used in all gene-centring heatmap and multivariate
visualisations. Additionally this matrix was transposed to get
sample-to-sample distances.

\subsection{Statistical analysis with generalised linear models (GLMs)}

The R-package edgeR (version 2.4.1) \cite{pmid19910308} was used to
build negative binomial generalised linear models, as these
specialised GLMs outperformed GLMs in DESeq in speed and reliability
of convergence. Modeled were based on a negative binomial distribution
and the dispersion parameter for each transcript was approximated with
a trend depending on the overall level of expression. In the maximal
fitted model expression was regressed on worm-sex, host-species and
parasite population, including all their interactions. The full model
thus contained terms $ S_i + H_j + P_k + (SH)_{ij} + (SP)_{ik} +
(HP)_{jk} + (SHP)_{ijk} + \varepsilon$, where $\varepsilon$ is the
residual variance, $S_i$ is the effect of the ith sex (male or
female), $H_j$ is the effect of the ith host species
(\textit{An. anguilla} or \textit{An. japonica}), $P_k$ is the effect
of the kth population (European or Asian), $(SH)_{ij}$ is the
sex-by-species interaction and similarly for the other interactions.

The hierarchical nature of generalised linear models was respected
considering (removing) all interaction effects of a main-term
(e.g. $(SP)_{ik}$, $(SH)_{ij}$ and $(SHP)_{ijk}$) when analysing
models for the significance of that term (e.g. $S_i$). Resulting
p-values were corrected for multiple testing using the method of
Benjamini and Hochberg \cite{benjamini1995controlling} and
differential expression was inferred at a false discovery rate (FDR)
of 5\% (adjusted p-value of 0.05).

\subsection{Count-collapsing for orthologs from two model-species}
\label{collapse}

In order to test the influence of deficiencies (i.e. fragmentation) of
the assembly on mapping we summed read counts over orthologous
sequence in \textit{C. elegans} and \textit{B. malayi}.  Differential
expression for these orthologous-counts was analysed the same way as
for contigs. Contigs were filtered based on inference from orthologous
counts merging the two orthologous evaluations and the contig
evaluation. Differential expression was accepted at a FDR of 5\% for
the contig evaluation and 10\% for both of the two orthologous
evaluations.

\subsection{Multivariate confirmation of linear models}

I used the R package vegan (version 2.0-2) to perform constrained
redundancy analysis on contigs identified as significant in GLMs
before. For each set of contigs (different for sex, eel-host or
-worm-population) the appropriate constrained component was used. The
proportion of the variance explained by the constrained component was
recorded and the constrained component was tested for significance
using a permutation test implemented in vegan.

\subsection{Gene ontology enrichment analysis}

Prior to analysis of GO term overrepresentation (based on dn/ds or
expression values) we used the R-package annotationDbi
\cite{AnnotationDbi} to obtain a full list of associations (also with
higher-level terms) from annot8r-annotations. We then used the
R-package topGO \cite{topGO} to traverse the annotation-graph and
analyse each node in the annotation for overrepresentation of the
associated term in the focal gene-set compared to an appropriate
universal gene-set (all contigs with dn/ds values or all contigs
analysed for gene-expression) with the ``classic'' method and Fisher's
exact test.

\subsection{Clustering analysis}

The R package HeatmapPlus was used on variance stabilised expression
values to visualise hierarchical clusters similar to the method of
\cite{pmid9843981}. The results were displayed along with annotations
stored in a Bioconductor eSet-class object.

\section{General coding methods}

The bulk of analysis (unless otherwise cited) presented in this paper
was carried out in R \cite{R_project} using custom scripts. We used a
method provided in the R packages Sweave
\cite{lmucs-papers:Leisch:2002} and Weaver \cite{weaver} for
``reproducible research'' combining R and \LaTeX code in a single
file. All intermediate data files needed to compile the present paper
are provided at For general visualisation we used the R packages
ggplot2 \cite{ggplot-book} and VennDiagram \cite{pmid21269502}.


\section{Results}

Dissection of eels 55-57 after infection (dpi) showed higher recovery
of the present day sympatric European worms in \textit{An. anguilla}
and higher recovery of Taiwanese worms in \textit{An. japonica},
compared to the allopatric host parasite combinations.

!!!Figure!!! Recovery of worms in coinoculation experiment. Mean
numbers of worms recovered after 55-57 dpi for sample sizes given as
n=x. Error-bars indicate the standard error (s.e.) of the
mean. Recovered lifecycle stages of the parasite are listed separately
as L3-larvae (l3), L4-larvae (l4), adult females (adult.f) and adult
males (adult.m).

In the sympatric host parasite pairs roughly eight or nine adult worms
and the same number of larval stged could be recovered per eel,
resultin in roughly 30\% recovery as a proportion of the 50
administrated larvae. In the transplanted host parasite combinations
only two or three adult worms were recovered on average and also the
number of larval stages was not higher (recovery below 10%). 

\begin{table}[h]
\begin{center}
\begin{tabular}{rrrrr}
  \hline
 & Estimate & Std. Error & t value & Pr($>$$|$t$|$) \\ 
  \hline
  (Intercept) & 9.5000 & 1.1109 & 8.55 & 0.0000 \\ 
  host.spec.AJ & -8.0789 & 1.7472 & -4.62 & 0.0000 \\ 
  worm.pop.T & -5.2222 & 1.3689 & -3.81 & 0.0002 \\ 
  host.spec.AJ:worm.pop.T & 11.7345 & 2.2010 & 5.33 & 0.0000 \\ 
   \hline
\end{tabular}

\caption[Linear model for recovery]{Linear model for recovery of adult
  worms. The estimate gives the mean of the distribution of adult
  worms for the factor values in the rows. The intercept is set to
  "Aa. R" (\textit{An. anguilla} and the European populations) further
  rows give variations for each factor. Std. Error is the standard
  error of this value. Additionally the probability of a t-value as
  small or smaller than the observed t-value are given. The signature
  of local adaptation is visible in the highly significant interaction
  term.}
\label{tab:ad-sig}
\end{center}
\end{table}

\section{Sample preparation and sequencing}

From three biological replicates data was obtained for each of 6
experimental groups: The two worm populations in each of the two eel
species and for each of the two sexes of worms. This resulted in a
total of 24 independent sequencing experiments.

!!!Table!!! Summary of RNA preparation. A summary of 24 samples
sequenced. The label of the RNA preparation follows a convention based
on the eel species (host; first two letter of label, AA for
\textit{An. anguilla} AJ for \textit{An. japonica}), worm population
(population - R for European, T for Taiwanese) and sex of worm(s) in
preparation (F for female, M for male; last letter in label). The
European samples were from two locations: river Rhine (R,) and
M\"uggelsee near Berlin (B), the Taiwanese samples were from from Kao
Ping River (K) and Yunlin county (Y). Additionally the intensity of
infection (number of adult worms found in the infected eel; intensity)
and the number of worms pooled in the preparation (only male worms are
pooled for RNA extraction, individual female worms were used). Finally
RNA-concentration in the preparation (conc in prep) is given in $\mu$g
per ml.

Sequencing was performed in three multiplexed pools of eight libraries
each. Each pool of eight was sequenced on two lanes, giving in total
six lanes of data and two technical replicates for each
library. Sequencing resulted in a total of 263,668,952 raw sequencing
read-pairs, each read having a length of 51 bases and 270 bases mean
insert size between the read pairs.

\section{Examination of data-quality}

Reads were mapped against the fullest pyrosequencing-assembly (!!! 454
paper !!!) using BWA \cite{pmid20080505}. Of the 263,668,952 raw
read-pairs 173,602,387 mapped uniquely to the assembly and were
counted on a per-library base.

The technical replicates demonstrated very low differences as inferred
from a clustering analysis using variance stabilised data and
transposed euclidean distances between samples (see figure ).

!!!Figure!!! Distances between RNA-seq read-count for different
samples. Euclidean distance (square distance between the two count
vectors) for variance stabilised read-counts for all libraries
including technical replicates; Red indicates low distance (high
similarity), blue high distance (low similarity). a) Data before
screening and summation of technical replicates. All technical
replicates are clustered very closely, the distance between an outlier
female sample (AJ\_T26F) is high. b) Same illustration after summation
of technical replicates and screening. Distance between outlier-sample
and other female samples is reduced.

158,232,523 read-pairs were left after removal of hits to contigs for
which non-\textit{A. crassus} origin had been inferred in the
analysis of the 454-transcriptome assembly.

!!!Table!!! Mapping Summary. Mapping is summarised for all 24
libraries. Rows indicate different libraries (worms or worm-pools as
indicated in !!!! raw.reads gives the number of read-pairs sequenced,
raw.mapped the number of reads mapping uniquely with their best hit,
tax.mapped the number of reads after subtraction of reads to putative
eel-host derived contigs and screened after subtraction of all reads
mapping not to the highCA-derived assembly or to contigs with overall
counts less than 32.

After another screening for spurious read-counts to low coverage
transcripts and to transcripts of low reliability (lowCA in the
454-assembly; !!!454paper!!!) 137,477,156 read-pairs were left for
further analysis. Distribution of these read-pairs over libraries
showed roughly 2.7-fold differences, with a mean of 5,728,215 reads
and a range from 3,422,526 read-pairs for library AJ\_R3M to 9,453,468
read-pairs for library AA\_R8F (see table XXX).

!!!Figure!!!Principle coordinate plot for expression in RNA-seq
libraries. Distance between sample-pairs is the root-mean-square
deviation (Euclidean distance) for the most differentially expressed
(DE) genes. Distances can be interpreted as the log2-fold-change of
the genes with the biggest changes, i.e. the log2-fold-change for the
genes that distinguish the samples.

These reads mapped to 7,520 contigs from our 454 assembly, making them
the basis for all further investigations.

In addition to hierarchical cluster analysis, also principal component
analysis grouped libraries according to the sex of worms (the largest
effect), but was unable to identify libraries with expression
correlated due to other experimental factors (eel host or parasite
population; see figure b). Between-sample distance confirmed the
hierarchical library clustering. Sex of the worms defined the overall
distances between libraries, host- or population-differences were not
visible in an overall effect in the top differentially expressed (DE)
genes (see figure). Amoung male samples distance was smaller than
among female samples.

\section{Orthologous screening for expression differences}

For the 7,520 contigs with expression values 4,382
\textit{C. elegans}-orthologs and 4,292 \textit{B. malayi}-orthologs
were determined based on the annotation of our pyrosequencing-assembly
(see cite !!!454-paper!!!). This resulted in 3,596 contigs with an
expression measurement, having a measurement also for both
corresponding orthologs (or group of orthologs) in both model-species
and thus being available for analysis.

For all further evaluations the congruence of the basic contig-based
statistics with orthologous-confirmed (OC) statistics is considered.

\section{Expression differences in generalised linear models}

Generalised linear models (GLMs) were used as implemented in the
R-package edgeR. Using these models we obtained 2,588 contigs (34\% of
total) DE between male and female worms at a false discovery rate
(FDR) of 5\%. 1,101 (31\% of total orthologous available) of these
contigs of were confirmed by contigs in the orthologous
evaluation. 1,425 (556 OC) of these were upregulated in male worms
1,163 (545 OC) in female worms.

At the same threshold, 55 contigs (0.7\% of total; 9, 0.25\% OC)
showed significant differential response to the host-species. 38 (5
OC) were upregulated in \textit{An. japonica}, 17 (4 OC) in
\textit{An. anguilla}.

68 contigs (0.9\% of total; 15, 0.42\% OC) showed differences
according to the population of the worm. 39 (11 OC) of these were
upregulated in the Taiwanese population, 29 (4 OC) in the European
populations.

An important observation in these models is the prevalence of
co-occurring significance of simple main effects. Expression changes
overlapping for two main effects mean a significant difference in
expression according to both factors. These differences are in the
same direction for a combination of the factors. Most contigs DE
according to the main effects of host-species or worm-population were
also DE according to the sex of the worm. There was also a number of
contigs differing for all three predictors in the same way. No contigs
were observed DE in both the host-species and worm-population in the
same direction but not according to worm-sex. From the 68 contigs DE
in different \textit{A. crassus}-populations, 38 were also DE
according to worm sex and 16 according to all three main effects (see
figure).

In addition, interaction-effects were also observed. In these
interactions a difference according to both focal factors in different
directions for factor combinations is indicated. For interactions
between host-species and parasite-population (eel/pop), for example,
this mirrors the result of adult recovery i.e. a differential
regulation according to sympatric host-species/parasite-population
combinations as found in nature: 7 contigs (0 OC) showed differential
expression according to the worm-sex/eel-species interaction, 12 (3
OC) to worm-sex/parasite-population, 13 (2 OC) to
host-species/parasite-population, 1 (0 OC) contig showed significance
for the 3-way interaction (see figure ). It should be noted, that
conclusions drawn from of simple main effects do not necessarily hold
for contigs with significant interaction effects (e.g. significantly
higher expression in European population can then mean higher values
only in one of the host-species).

!!!Figure!!! Venn diagram of contigs significant for different terms
in edgeR GLMs. Overlap between differences in simple main effects are
given as black numbers in the Venn-Diagram. Numbers outside the
circles in the lower left corner indicated non-significant
contigs. The number of significant contigs for interaction effects are
indicated in red for comparison. In (a) values for all contigs are
given in (b) for ortholog-confirmed (OC) contigs.

In summary, a low amount of overlap in main effects between
populations and host-species compared to the other main-effect
overlaps and in relation a higher proportion of interaction effects
between these two conditions was observed.

\section{Confirmation of contig categories through principal component
  analysis}

I performed constrained redundancy analysis for the effects of
eel-host and worm-population. This technique, similarly to principal
components analysis, can partition the variance into orthogonal
components, and additionally constrain one of the components to the
factor of interest. I found that 7\% of the variance in contigs DE
between eel-hosts and 11\% of the variance in contigs DE between
worm-population explained by the corresponding factor. In both
evaluations more than 50\% of the remaining variance could be
explained by a single principal component, to which sex contributed
over 99\% (loading) (see figure and ). When only OC-DE contigs were
considered the explained variance for difference between eel-host
dropped to 3.3\% and the explained variance for differences between
worm-population was raised to 23\%, while the sex-effect explained
70\% and 50\% of the variance (see figure and ). Significance of the
constrained component evaluated by a permutation-test could be
established at a p $<$ 0.05 threshold for all but the OC eel-host DE
subset.

!!!!Figure!!! Constrained redundancy analysis for host-DE
contigs. Eel-host differences are displayed as constrained component
on the x-axis, the sex contributed >99\% (loading) to the principal
component on the y-axis. (a) Host differences partition the variance
in samples in like expected for all contigs, the constrained component
showed significance. (b) For OC contigs the constrained component
fails to to partition the variance as expected, the component showed
no significance for this subset of the data.

!!!Figure!!! Constrained redundancy analysis for population-DE
contigs. Population differences are displayed as constrained component
on the x-axis, the principal component on the y-axis corresponds to
the sex of the worm. Host differences partition the variance in
samples like expected for all contigs (a) as well as for OC-contigs
(b). The constrained component showed significance in both subsets.

\section{Biological processes associated with DE contigs}

I employed tests for overrepresentation of categories in
gene-ontology (GO). These tests respect the structure of the ontology
and also consider overrepresentation of higher level (ancestor-)
terms. Summarising annotations at higher levels it is therefore
possible to conceive higher-order responses to the conditions
investigated.

For the differences between male and female worms enriched annotations
can be summarised into three broad categories: Terms overrepresented
due to spermatogenesis (e.g. PP1-phosphatase and ester hydrolase are
important for spermatogenesis in \textit{C. elegans}
\cite{wormbook_sperm, fardilha2011protein}) embryo development (many
obvious terms) and terms for other processes more related to metabolic
differences between males and females (such as oxidoreductase
activity; see table XXX but also additional figures and).

For the lower number of contigs DE between host-species inference of
higher order terms was obviously only possible to a limited extent and
in part also unnecessary, because annotations can be interpreted at
face value. However, annotations for contigs DE between eel-hosts
highlighted redundant terms associated with ``antigen processing and
presentation'' proteins which are in mammals usually involved in
antigen processing and cleavage of the invariant chain of the MHCII
complex. These terms led to Contig566 and Contig26 and their
\textit{B. malayi}-orthologs ``aspartic protease BmAsp-1, identical''
and ``eukaryotic aspartyl protease family protein''. In blood feeding
helminths these enzymes are in contrast usually involved in early
cleavage events during the digestion of host haemoglobin
\cite{pmid12782060}.


!!!Figure!!!GO biological process graph for enriched terms in DE
according to worm-population. Subgraph of the GO-ontology biological
process category induced by the top 10 terms identified as enriched in
DE genes between different parasite populations. Boxes indicate the 10
most significant terms. Box colour represents the relative
significance, ranging from dark red (most significant) to light yellow
(least significant). In each node the category-identifier, a
(eventually truncated) description of the term, the significance for
enrichment and the number of DE / total number of annotated gene is
given. Black arrows indicate a is ``is-a'' relationship.

For contigs DE between worm populations despite the limited number of
DE contigs, enrichment analysis identified ``oxidoreductase activity''
as an informative significantly enriched higher level term (see
figure). The biological processes ``response to metal ion'' and
``mitochondiral electron transport'' (see figure) confirmed an
evaluation linking these mainly to enzymes used in respiratory
processes and highlighted additionally enzymes from lipid metabolism
(especially $\beta$-oxidation of fatty acids) related to respriration
and the availability of oxygen.

\section{Clustering analysis}

For the remainder of the text I will concentrate on these differences
of the European and Taiwanese populations and mention the other
differences only as far as they are related to this focal factor. In
figure however, graphical analyses of the same type are presented for
other factors.

Clustering analysis uses distance measurements between samples as well
as genes (or transcripts) to highlight patterns of similarity. The
classical distance measure used in hierarchical clustering throughout
this document is Euclidean distance. Grouping of genes regulated in
parallel in combination with annotation, the status of cellular
processes can support notions based on single genes.

Hierarchical clustering analyses of genes DE between populations
confirmed the results of principal component based multivariate
analysis. The main factor grouping libraries was the sex of the
worm. A sub-grouping of samples fully according to European and
Taiwanese populations was only observed for male worms. In female
worms other unmeasured co-factors were preventing a clustering fully
according to this factor. In male worm however, library clustering
even followed a pattern of similar expression in according to the
second factor of eel-host. These statements are true for both the full
set of contigs (see figure ) and OC contigs (see table XXX).

Clustering of genes revealed three co-regulated groups in the full set
of contigs and the OC set. The first of gene-clusters (top in
\ref{pop_all_heat} and \ref{pop_ortho_heat}) was in sex-subgroups
mainly following an expression pattern differing between
populations. The second gene-group was much larger in the full set
than in the OC set of contigs (middle in \ref{pop_all_heat}). It was
only very weakly reacting to any other factor but sex and was very
sparsely annotated (therefore this group was much smaller in the OC
set \ref{pop_ortho_heat}). The third gene-group found again in both
the full and OC contigs (bottom in \ref{pop_all_heat} and
\ref{pop_ortho_heat}) was reacting on both the host and population
factor in a converse way. Contigs in this cluster were mainly found to
be significant for interaction effects.

!!!Figure!!! Clustering of expression values for contigs DE between
populations. A heatmap of variance/mean stabilised expression
values. Deprograms are based on hierarchical clustering. Green
indicates expression below the mean, red above the mean. Experimental
conditions are indicated by black bars for groups of samples (columns)
below the plot. Presence GO-term annotation for contigs (rows) are
given as black bars right to the plot: isOxidoreductase = GO:0016491,
oxidoreductase activity; isMitochondrial = GO:0005739, mitochondrion;
isELDevelopment = GO:0002164, larval development or GO:0009791,
post-embryonic development; isResponsetoStim = GO:0050896, response to
stimulus; isPhosphatase = GO:0016791, phosphatase; isMembrane =
GO:0016020, membrane; isAntigenProc = GO:0002478, antigen processing
and presentation of exogenous peptide antigen; isEndosome =
GO:0005768, endosome; isProtLipComp = GO:0032994, protein-lipid
complex. Grey bars indicate no annotation available.

Consolidating the clusters with annotation and annotation-enrichment,
the first cluster of genes was very well annotated and contained
mostly catalytic enzymes involved in oxidation and reduction, the
bottom cluster contained more unannotated genes and structural
(cuticular collagen) genes.

\section{Single gene differences}
\label{sec:single-gene-diff}

Tables on single transcript values of OC contigs DE between eel-hosts
and populations can be found in additional tables XXX and
XXX. Obviously for some contigs differences significant in the model
are rendered inaccessibly by comparing simple mean values because of
superposed interaction effects or overwhelming general effects of worm
sex.

Cytochrome C oxidase subunit 2 (COXII) shows the clearest of all
expression patterns for any of the observed genes. It differed
significantly only between populations (showed no reaction an any
other factor) and was on average over 1,000-fold stronger expressed in
the Taiwanese population. At face values differed for every single
individual (of the 12 investigated in each populations) at least
20-fold (highest normalised expression was 350 counts in a European
worm, lowest normalised expression in any Taiwanese worm was 7,500
counts). Counts summed for orthologs were also significant only for
this factor and showed over 10-fold stronger expression in the same
direction. This accounts to the fact, that misassembled contigs
containing fragments of COXII were only adding experimental noise.

\section{Discussion}


% It is one of the dangers of genomic data to forget the fundamental
% lesson from the debate initiated by Gould and Lewontin in 1979
% \cite{gould_spandrels_1979}. Briefly, while functional changes are
% often caused by selection, differences in function do not necessarily
% demonstrate the past or present action of selection. There is no way
% to infer the action of selection based on functional considerations,
% and even if selection can be inferred otherwise, it is not necessarily
% a particular observed variable trait that selection acted on
% \cite{pmid19744124}.

\subsection{Recovery and adaptation}
\label{sec:recovery}

With some reservations discussed below observation of higher recovery
of adult worms from sympatric \textit{A. crassus}-\textit{Anguilla}
spp. host-parasite combinations imply local adaptation of different
worm populations to host species: Roughly one third of the applied
European worms were recovered form their sympatric host
\textit{An. anguilla} but only little over 10\% for European worms in
\textit{An. japonica}. This pattern of recovery was precisely inverted
for the Taiwanese population of \textit{A. crassus}, for which
recovery was thus roughly 30\% in the sympatric \textit{An. japonica}
and only 10\% in \textit{An. anguilla}. 

Data for the European eel are in agreement with
\cite{knopf_differences_2004} and !!!Weclawski!!!, who did however not
find lower recovery of the European populatio in the Japanese eel.

An ideally suited phenotype to infer local adaptation in a common
garden experiment would be one with or resulting from direct fitness
consequences, a so called fitness component. Fitness is defined as the
differential contribution to the next generation, therefore such a
fitness component would ideally be a measurement on a single
individual, and individual life time reproductive success would be an
ideal measurement. However, techniques to measure such individual life
time reproductive success have not been established in
\textit{A. crassus}.

The recovery of certain developmental stages of worms is only a proxy,
interpretable as a fitness component. It is a composite measurement of
the speed of development from previous lifecycle stages (or speed of
migration towards the swimbladder) and of survival. While survival is
surely an important component of fitness, it is not completely clear
whether fast development and/or migration to the swimbladder are. It
is possible that under certain conditions slower development could
lead to higher fitness, if it would, for example allow development
without attracting the attention of the immune system.

\subsection{Divergence not modification of expression}
\label{sec:sample-twelve}

I decided on a study design using pools of individuals for one sex
(males) and single individuals for the other. A study on
\textit{Fundulus heteroclitus} revealed that approximately 18\% of the
transcripts are differentially expressed between individual fish from
the same population, grown under controlled environmental conditions
\cite{pmid12219088}. And it thus not surprising that between
individual variation in female samples was leading to higher variance
of these female samples compared to pooled male samples in my study.

This interindividual variation in gene expression under a particular
environmental condition is generally agreed to be closely linked to a
genetic basis \cite{pmid15498452}. For example in a cross between two
parental strains of yeast the genetic component of variation was
estimated from haploid segregants to be 84\% \cite{pmid11923494}. The
genetic component was found to be the main factor determining
expression level variability between two strains, sexes and ages of
\textit{Drosophila melanogaster} for 267 (7\%) from 3,931 genes and at
least 25\% of the transcriptome were estimated to be affected mainly
by genotypic factors in any of the groups
\cite{pmid11726925}. Variation in the regulation of gene expression is
thought to constitute a major source of evolutionary novelty
\cite{pmid11341673}.

A second study from the line of research on \textit{Fundulus
  heteroclitus} \cite{pmid16567645} used genetic relatedness as
inferred from phylogenetics to separate variation in gene expression
in a common experimental environment into a neutral component and a
selected component, this way removing variation most likely accounted
for by the shared neutral evolutionary history. My case of
\textit{A. crassus} is potentially simpler: the investigated European
populations are direct descendants and thus a subset of a Taiwanese
source population. In fact I studied two European and two Taiwanese
populations as a few hundred kilometers between the geographical
origins of the two different locations in Germany and Taiwan probably
constitute a barrier to gene flow in a parasite with an aquatic
intermediate host. However, I treated worms from both European and
Taiwanese populations as replicates (and use the terminology of one
European and one Asian population throughout the text) with the
rationale of increasing variance for random genetic differences and
raising the bar for potentially adaptive differences to be detected.


Gene expression values constitute nothing more than a molecular
phenotype. This phenotype is not necessarily a fitness component.

Given the sampling of only twelve Taiwanese worms the question could
be raised, whether these constitute a representative sample of the
true source population, of which a subpopulation was funding European
populations. A microsatellite study indicated gene flow even between
populations of \textit{A. crassus} separated by thousands of
kilometers in Asia (Japan and Taiwan)
\cite{wielgoss_population_2008}. Given the high interconnectivity of
Taiwanese water systems used for aquaculture both by man build
structural links and anthropogenic exchange of fish, a sampling from
two Taiwanese populations similarly neutrally diverged from the true
European funding population seems very unlikely. The worms sampled
from Taiwan can thus be regarded a sample of the metapopulation
appropriate for finding differences in relation to the source of the
introduction.

Of no surprise was the abundance of differential expression between
male and female worms in roughly one third of the genes. A large
number of genes are known to be sex specific, regulating ovulation and
spermatogenesis throughout the metazoa and especially in nematodes
\cite{pmid15371532}. On top of these sex specific genes there are
large numbers of genes differently expressed due to differences in
metabolism between males and females. Estimates for
\textit{Drosophila} based on similar sample sizes to those used in my
study range between one and two thirds of the transcriptome showing
sex biased expression \cite{pmid11726925}. In the liver transcriptome
of \textit{Mus musculus}, even 70\% of transcripts have been shown to
differ between sexes \cite{pmid16825664} (note however that this study
used 169 female and 165 male mice to guarantee the finding of even the
most subtle differences). Given the scale of these differences in
other species my estimate of roughly one third of the transcripts in
\textit{A. crassus} showing differential expression according to the
sex of the worms implies conservative thresholds used in the
statistical analysis and moderate power for detection of differences.

Nearly the same proportion (roughly 30\%) of contigs was confirmed
through summation and analysis of contigs for orthologs in
\textit{B. malayi} and \textit{C. elegans}. Development of this
orthologous confirmation method was necessitated by the possibly
fragmented and chimeric transcriptome assembly. This introduces
stringent conditions for the detection of significance, as p-value
correction for multiple testing is employed during each analysis (once
for raw counts and twice for orthologous counts). Although the
underlying tests are not independent, the false discovery rate of 5\%
for raw contigs can be expected to be immensely lowered by applying a
FDR of 10\% twice.

In addition biological implications could produce false negatives in
such an evaluation: All genes duplicated in \textit{A. crassus} (a)
and following antithetic expression patterns will be evaluated
negatively, as will duplicated genes in any of the model species (b)
following such a pattern. However, there is no other choice then
applying these stringent conditions to screen for artefacts producing
the same patterns based on mapping to fragmented (a) or chimeric (b)
reference contigs. I think that an evaluation based on this
scrutinised confidence in an assembly previously computed from 454
data is even more appropriate then an analysis solely based on counts
collapsed for orthologs excluding only possible fragmentation
artefacts (as used e.g. in \cite{pmid22084086}).

In general, my statistical analysis aimed to minimise false positives
(type I error) at expenses of possible false negatives (type II error)
and is thus not fully suited to address the proportions of
differentially regulated genes.

Nevertheless it is surprising that less than 1\% of transcripts were
detected differentially expressed between worms in different host
species and less then 0.3\% were confirmed with the orthologous
summation method. This was an unexpected finding, as the differences
in the immune response of the host species have a big influence on
other phenotypes of worms \cite{knopf_swimbladder_2006}. In addition
to the low number of genes, multifactorial analysis revealed that
below 10\% of the variance could be explained by host species effect,
even in significantly differential regulated genes for this factor.

Although these differences between worms in different host species
were the most marginal of any of the factors, it is possible to
connect some (at least two) of the genes to a prominent physiological
difference: the digestion of haemoglobin. Two different aspartic
proteases (both confirmed through orthologs, one of them differing for
all three main effects, the other for an interaction of worm sex and
host species) known to be involved in the first steps of digestion of
haemoglobin from other nematodes \cite{pmid12782060} were
overexpressed in worms in \textit{An. anguilla}. This expression
phenotype could potentially be linked to the often observed phenotype
of bigger size of \textit{A. crassus} in this host
\cite{knopf_swimbladder_2006}, as the main contribution to this
increase in size is the larger volume of host blood taken up by the
parasite. Accordingly the parasite probably digests haemoglobin at a
higher rate.

Close to 1\% of contigs were significantly different in expression
between European and Asian \textit{A. crassus}, making this difference
significant for a higher number of contigs than the host
differences. For this contrast the proportion of orthologous
confirmation was lower than for sex differences but higher than for
host species differences. Additionally multivariate analysis of all
differently expressed transcripts for worm population revealed that
the variance contributed by the population factor was higher than 10\%
for all significant contigs or even 20\% for orthologous confirmed
contigs.


The benefit of also allowing contrasting significant differences in
interaction terms highlights the power of the GLM-approach.

Another important finding was the large overlap in contigs expressed
differentially depending on worm sex and worm population. Such an
overlap is expected if genes expressed differentially according to sex
are evolving faster towards a differential expression according to
other factors. Faster evolution of reproductive (and especially male
specific) traits has been shown in many species at a phenotypic and at
a sequence level \cite{pmid15795858}. In \textit{Drosophila}, male
reproductive proteins have been shown to evolve at elevated levels and
under positive selection \cite{pmid11404480}. Moreover, gene
expression should evolve at a higher rate in sex specific genes.
Indeed the transcriptomes of \textit{Drosophila} species show that
interspecific expression divergence is sex dependent and the action of
sex dependent natural selection during species divergence has been
inferred from this \cite{pmid15034135,pmid19720861}.

Taken together, my findings strongly support a stronger influence of
genetic differences between European and Asian populations of
\textit{A. crassus} than of the modification in the different host
species on gene expression. When additive and interaction effects are
considered, the influence of host species even vanishes almost
completely in favour of a combination of effects combining parasite
population and sex of the worms.

\subsection{Functions of genes with genetically fixed expression
  differences}
\label{sec:function-genes-with}

From a functional perspective, genes identified to differ between
populations can be categorised as important in general metabolic
processes instead of specific host parasite interactions. This
constitutes a negative evaluation of one of my \textit{a priori}
hypotheses based on finding parasite specific genes, identified as
vaccine candidates in a number of nematodes, within the genes modified
or diverged in my study (\ref{sec:dna-sequ-nemat}). However, more
direct host parasite interactions are expected in tissue dwelling
larval stages (L3 and L4) and in fact most immunomodulators are
expressed predominantly in these stages
\cite{maizels_helminth_2004}. Adults of \textit{A. crassus} could thus
be the wrong lifecycle stages to detect such expression differences,
if they existed.

\subsubsection{Metabolism}
\label{sec:metabolism}

Instead enzymes and enzyme subunits important for aerobic respiration
are especially expressed at lower levels in European
\textit{A. crassus}. In fact, most transcripts significantly differing
between populations were annotated as ``oxidoreductase'' in gene
ontology (GO).
% While at face values of single genes have to be interpreted with
% care such trends in the data are very likely to point to important
% differences.
Downregulation of cytochrome C oxidase subunit 2 (COXII) in the
European population of \textit{A. crassus} was the most persistent
finding. This downregulation was confirmed by the low expression of
the same contig in the European libraries compared to higher
expression in all three libraries from Taiwanese worms in
pyrosequencing. Cytochrome C oxidase subunits 1-3 are are essential
components of respiratory chain complex IV, the cytochrome c
oxidase. They are encoded in the mitochondrial genome and coordinate
catalytic heme and copper cofactors \cite{pmid18023115}.

In fact, not only enrichment analysis highlighted oxidoreductases, but
expression values of COXII clustered with other enzymes related to the
state of energy metabolism: two lecitin:cholesterol acyltransferase
transcripts are putative recently duplicated genes. They showed
slightly divergent protein sequences but hit the same orthologs in
\textit{C. elegans} and \textit{B. malayi}. They also shared very
similar expression profiles. Expression of different cholesterol
acyltransferases has been shown to vary in response to the presence of
heme and anaerobiosis in yeast \cite{pmid11786267}. 3-hydroxyacyl-CoA
dehydrogenase (involved fatty-acid $\beta$-oxidation
\cite{pmid8454629}), malate/L-lactate dehydrogenase (from the
anaerobic glycolytic pathway or the Krebs-cycle
\cite{sturm1969vergleichende}) and aspartyl proteases (involved in the
digestion of host haemoglobin in helminths \cite{pmid12782060})
completed this particular cluster.

These patterns can be interpreted as a biological confirmation of the
at face values for single genes, especially for COXII. In addition the
differential reaction of metabolic genes to different factors (genetic
vs. modification) invites speculation on a causal structure behind
these correlations. The expressions of metabolic enzymes are
interpretable as a change to use a more anaerobic metabolism in the
European population of \textit{A. crassus}. In one possible scenario,
in European worms one of the subunits of core enzymes of the
respiratory chain (probably COXII) would have evolved a genetically
fixed lower level of expression. This model follows the logic that the
most differential expressed gene could be the driver of observed
change. Other enzymes related to aerobic energy metabolism directly or
indirectly via the redox state of cells (e.g. lipid metabolism) and
only partially controlled by feedback mechanisms from oxidative
phosphorylation and the citric acid cycle would show similar patterns
of altered expression in European worms. However, the expression of
these indirectly and also by additional environmental factors
controlled genes would be perturbed when worms are applied back to
their Asian hosts. Also in the two sexes differences in size and
metabolism would be perturbing the pleiotropic effects of the
persistent core change.

Such a scenario also provokes speculation about the adaptive value of
such a change in a core metabolic process: aerobic respiration is a
potential source for oxidative stress providing a steady source of
reactive oxygen species (ROS) as electrons are leaking from the
respiratory chain as superoxide anions. It is well established that
such ROS production is especially harmful to blood feeding parasites,
as free inorganic iron, as well as heme, have the potential to
generate additional ROS \cite{pmid21087517}. Anaerobic metabolism is
thus thought to occur in many haematophagous parasites as a counter
measure against oxidative stress from haemoglobin catabolism
\cite{pmid12163151}. It could thus be hypothesised that the bigger
size and the larger amount of eel blood ingested leading to a higher
rate of haemoglobin digestion provided the selective pressure to
reduce aerobic respiration. Additionally helminths can simply get too
large to maintain oxygen diffusion to mitochondriae in the absence of
a cardiovascular system. As yet proton pumping electron transport
constitutes the most profitable energy providing process, the
mitochondriae of facultatively anaerobic helminths produce a proton
gradient for the use of ATPase with the help of terminal electron
acceptors other than O$_2$ \cite{pmid12417132}. Such an alternate
electron sink is fumarate used in many helminths in a process called
malat dismutation \cite{pmid15275412}.

An interesting implication is that such metabolic differences could
potentially be visible ultrastructurally. Indeed in my own diploma
thesis \cite{heitlinger_vergleichende_2008} I identified two different
kinds of mitochondriae, one with standard christae like morphology,
the other with unusual sacculus like morphology in
\textit{A. crassus}. Additionally I observed less electron dense
inclusions (probably lipid reserves) in bigger worms and more glycogen
granulae. The fact that such lipids are less usable under anaerobic
conditions led me to the hypothesis that bigger worms are using less
aerobic processes. Reanalysing this data and probably obtaining new
data with additional histochemical staining methods could be a way to
put gene expression into a physiological perspective. Furthermore, a
biochemical examination of isolated mitochondriae could highlight
changes in the mitochondrial respiratory chain under \textit{in vitro}
conditions \cite{pmid18314717}. Such direct measurements of COX enzyme
activity (using well established assays \cite{pmid8592440}) would be
desirable to establish even the validity of the first logical step in
these adaptive speculations that underexpression of COXII is leading
to decreased enzyme activity. It would be counterintuitive to expect
higher enzyme activity when COXII mRNA levels are low, but, for
example, in \textit{Schistosoma mansoni} COXI over expression in
praziquantel resistant strains is leading rather to decreased enzyme
activity \cite{pmid9695101}.

The sensitivity to perturbation of mitochondrial genes for respiratory
chain complexes in nematode parasites is underlined by their
upregulation after depletion of Wolbachia from filarial nematodes
\cite{pmid20362581, pmid19806204}. Wolbachia are obligate endosymbiont
bacteria of some clade III nematodes, they are supplying heme to
non-haematophagous parasites in the absence of an intrinsic pathway
for heme synthesis \cite{ghedin_draft_2007} (which is absent also in
free living \textit{C. elegans} \cite{pmid15767563}). While my
sequence analysis suggests the absence of wolbachial symbionts in
\textit{A. crassus}, such studies support a central role of host or
endosymbiont derived heme for respiratory processes and suggest a
propensity for evolutionary change in related processes (in Filaria
even acquisition of an endosymbiont).

Assuming a genetically fixed lower expression of COXII in European
\textit{A. crassus} as a driver for other metabolic differences does
not imply a simple regulation of the expression itself, or a
genetically simple change underlying the changed expression
phenotype. Regulation of the mitochondrially encoded genes has been
extensively integrated into the regulatory network of eukaryotic
cells and is controlled by and interacting with nuclear transcription
factors \cite{pmid8289797}.

Intriguingly overexpression of respiratory chain enzymes was limited
to cytochrome c oxidase transcripts (COXII and to lesser extent also
COXI and COXIII). Mitochondrial transcription produces multiple
polycistronic unmatured transcripts, which are cleaved and modified in
their expression post-transcriptionally. Cleavage occurs at t-RNA
sequences interspersed between protein coding genes and can be
imperfect to leave some transcripts polycistronic in a matured
state. Nevertheless, due to posttranscriptional modification
individual transcripts can be expressed uncoordinated, even when
expressed on the same unmatured polycistronic transcript
\cite{pmid19843606}. The addition of poly-A tails, for example, is
vital for stability of mature transcripts in metazoans. The
mitochondrial genome contains only very little untranscribed sequence,
is polyploid (once homoplasmic, essentially maternally inherited like
haploid) and transmitted completely linked, with very scarce
recombination events \cite{pmid18023115}.

Cis-regulatory change in a control region would thus be very easily
detectable in my transcriptome data. Even if the sequence variation
leading to the observed expression phenotypes would locate to the
untranscribed hypervariable mitochondrial control region (in D-Loop
associated promoters), selection on such a variant would render the
whole mitochondrial genome inadequate for phylogenetic analysis, as a
variant sweeping to fixation would have removed polymorphism from the
complete mitochondrial genome due to the prefect linkage
\cite{pmid19821901}. If a sweep would be presently ongoing, high
levels of heteroplasmy would be found in single individuals
\cite{pmid21226948}. Such a pattern has not been found in populations
of \textit{A. crassus} in Europe when COXI was used as a marker
\cite{wielgoss_population_2008, dl_py} (see also figure) and is also
not visible from preliminary analysis of polymorphism in mitochondrial
genes in my RNA-seq data.

Functional constraints are also expected regarding the mechanism by
which the expression of COXII could evolve. Most infective L3 larvae
of parasitic nematodes rely on aerobic respiration
\cite{kennedy2001parasitic}. Dixenous parasites like
\textit{A. crassus} migrate through tissues of definitive hosts, where
oxygen is readily available, after leaving the haeomocoel of the
intermediate host. Enzyme subunits building a functioning aerobic
respiratory chain are thus likely to be expressed at earlier lifecycle
stages of \textit{A. crassus} and elevated anaerobiosis is expected to
be restricted to the adult stages.

These considerations make sole or predominant cis-regulatory change in
mitochondrial DNA unlikely to explain the divergent expression
phenotypes. Still identification of the genetic architecture, for
example sequence variation in a transcription factor, a co-factor or a
protein modifying mitochondrial transcripts, may be possible (to a
limited extent even in the present RNA-seq data).

RNAi screens in \textit{C. elegans} for increased lifespan focus on
genes leading to lower oxygen consumption and altered mitochondrial
morphology and function \cite{pmid12447374}. Such candidate genes will
provide an additional link back to functional considerations once
screening for genomic regions with signature of selection will
highlight candidate loci.

% Numts should normally be avoided by reverse transcription, but
% occasionally Numts are transcribed

% (Blanchard, J.L. and Schmidt, G.W. (1996) Mitochondrial DNA migration
% events in yeast and humans: Integration by a common end-joining
% mechanism and alternative perspectives on nucleotide substitution
% pattern. J. Mol. Evol. 13, 537–548).

% Pseudo-mRNAs shadowy entities that resist classification and analysis
% resemble protein-coding mRNA, but cannot encode full-length proteins
% owing to disruptions of the reading frame.  \cite{pmid16683022}

\subsubsection{Collagens}
\label{sec:collagens}

A second group of genes differentially expressed in populations of
\textit{A. crassus} emerged from both cluster and enrichment
analyses. Two transcripts in this cluster were significant for
interaction effects between host species and parasite-population, they
were annotated as collagens. For both genes this meant an ``adjusted''
(to avoid the suggestive ``adapted'') expression difference leading to
a lower expression in sympatric host species/parasite population
pairs. Cuticle collagens are a large multigene family (Interpro lists
164 entries for ``Nematode cuticle collagen, N-terminal'' for
\href{http://www.ebi.ac.uk/interpro/ISpy?ipr=IPR002486&tax=6239}{\textit{C. elegans}}
and 51 for
\href{http://www.ebi.ac.uk/interpro/ISpy?ipr=IPR002486&tax=6279}{\textit{B. malayi}}),
containing extensive repeat regions: roughly 50\% Gly-X-Y residues,
often Gly-Pro-Hpy. In the genome of \textit{B. malayi} 82 genes
encoding collagen repeats have been found \cite{ghedin_draft_2007}. It
was thus very important to have orthologous confirmation for these two
contigs, as misassembly could have easily lead false positives here.

The two collagens were clustered with a third contig sharing a
collagen annotation (failing to be significant for the interaction
term probably because of low overall expression) and a contig
annotated as ``Matrixin'' (a metallo-proteinase assumed to be involved
in remodelling of the extracellular matrix \cite{mealloprot}) and a
ABC-transporter family protein.

Functional speculations are more difficult for collagen than for the
respiratory chain enzymes. The cuticle constitutes an exoskeleton and
a barrier between the worm and its host environment. Synthesis of most
collagens is believed to occur at negligible levels in adult male
worms and is rather constrained to discrete temporal periods in larval
development, the moults \cite{pmid10637627}. The differential
expression could thus be due to changes in larval development or due
to alternations in the low level, steady renewal of the adult cuticle
and remodelling of the extracellular matrix of hypodermis cells. Some
considerations would favour of the second explanation: in
\textit{C. elegans} genes expressed after reproductive maturity evolve
faster than genes expressed earlier in development
\cite{pmid15371532}. This suggests a model of elevated pleiotropic
effects in genes expressed at earlier stages of development and hence
more conserved expression patterns in larval stages. Independent of
these considerations, both the primary assembly and the constant
remodelling of the cuticle involve complex post-translational
processes hardly accessible at the transcriptomic level: a zipper-like
nucleation/growth mechanism leads to the folding of a triple helix of
and heterotrimers and homotrimers \cite{kennedy2001parasitic}. If and
how differential expression of two particular collagens interferes
with this process requests further research. As for the metabolic
differences, differential expression patterns could be reflected in
morphology. One approach would be to measure thickness and density of
the cuticle of worms from coinoculation experiments.

\section{Outlook}

The presented project on the divergence of gene expression obviously
constitutes work in progress. The observed differences in subunits of
respiratory chain enzymes, especially in COXII, necessitate and permit
confirmation by reverse transcription quantitative PCR (RTqPCR) for
these transcripts. Such evaluations of a single gene (or few genes)
will be possible on many individual specimen of \textit{A. crassus}
from both Europe and Taiwan to further test the significance of the
observed differences. Therefore, in addition to the validation of
expression values for sequenced samples, many of the worms from the
presented coinoculation experiment yielding lower amounts of RNA
inadequate for sequencing will be used to further establish the
divergence in gene expression. Additionally sampling of worms from
their present day sympatric hosts is possible for genes differing only
for populations unconditional on eel host species. Moreover, if
selection in Europe would have acted on standing variation, one would
expect to find worms expressing for example COXII at low levels also
in the Taiwanese source populations, at least in low frequency.

An assembly of the mitochondrial genome of \textit{A. crassus} from
preliminary genome-sequencing data (discussed below) and the
identification of the poly-cistronic unmatured and, if present,
matured transcripts (similar to \cite{pmid19843606}), will further
inform and validate the analysis of the expression of mitochondrial
genes. Additionally, disentangling assembly artefacts complicating
mapping from real nuclear or even mitochondrial \cite{pmid20026478}
pseudogenes of mitochondrial genes will help increasing the power of
expression analysis and furthermore permit the analysis of interaction
of such pseudogenes with the expression of functional genes.

Multiple starting points also exist for further functional examination
of metabolic change, as mentioned throughout the text. However, the
search for ultimate causes for evolutionary change \textit{sensu}
\cite{mayr1961cause} will potentially be even more rewarding.

I will expand the RNA-seq analysis presented here to study
allele-specific expression and the association between gene expression
and sequence variants. This kind of quantitative expression trait
locus (eQTL) analysis is possible as both sequence and expression
information are available from the present RNA-seq data. Both simple
cis-acting variation in promoter or enhancer regions, as well as
trans-acting variation can theoretically be detected
\cite{pmid21838806}. To detect trans-acting variants, however, might
be impossible with the (for population studies) relative low number of
sequenced individuals, as it relies on statistical associations
requiring broad sampling. Yet, cis-acting variation, more readily
detectable as allele-specific variation, is unlikely to explain
variations in mitochondrial gene expressions for the reasons discussed
above.

Therefore, large scale meta-population wide sampling must not be
limited to an evaluation of the divergent gene-expression phenotypes,
but has to further elucidate the population genetic relationships
between Taiwanese and European worms. A future research program will
thus need to employ population-scale sampling of genotype data, densely
spread across the genome. Genotyping of many European
\textit{A. crassus} from different populations and comparison with
many individual genomes from different Asian populations will enable
tests for selection: based on the fact that around selected variants
nucleotide diversity is reduced by hitchhiking of neutral variation in
so called selective sweeps \cite{pmid16251466}, a punctual increase of
population differentiation measured by the fixation index F$_{st}$
\cite{wright1949genetical} in regions linked to selected variants can
be measured. Other well established population genetic measurements
include Tajima's D, a measure based on the allele frequency spectrum
\cite{pmid2513255}. When these methods are applied on a genome wide
scale the neutral null-expectation to separate a loss in variability
based on selection from neutral loss due to demography is given by the
diversity across all regions of the genome. A microsatellite study
\cite{wielgoss_population_2008} as well as my own evaluations (based
on pyrosequencing see \ref{sing-w}) and RNA-seq (data not shown)
indicate only a moderate genetic bottleneck caused by the introduction
of \textit{A. crassus} to Europe and thus the necessary neutral
diversity as a background for these tests will be present.

Furthermore statistical models need to be parameterised by divergence
time to disentangle the influence of demography and selection (i.e. to
estimate the effective population size). Reliable estimates for
divergence time are readily available for the introduction of
\textit{A. crassus} to Europe: 60 to 90 generations. As for such a
short period linkage to putatively selected variants will not be
broken down in large blocks, marker density is of minor concern, but
priority should be given to the breadth (many individuals from many
populations) of sampling.

One methods enabling such population wide genotyping emerging from NGS
technology is the sequencing of restriction-site associated DNA (RAD)
markers. Preparation of RAD libraries involves digestion of genomic
DNA with a restriction enzyme. Individually tagged adaptors can then
be ligated to the fragments and individual samples can be pooled. The
choice of restriction enzyme is important to optimise the number of
restriction sites (depth of sampling the genome) relative to the
number of individual samples being investigated
\cite{pmid18852878}. In the case of \textit{A. crassus} this
optimisation also concerns the minimisation of restriction sites in
host-genome, as present in unavoidable contamination. 

The \textit{de novo} assembly of a reference genome for
\textit{A. crassus} will enable the search for such an optimal
restriction enzyme. Preliminary data has been generated for a female
individual of the Polish population on one lane of the Illumina HiSeq
machine, giving 110 million 100 bases long paired-end reads, in total
over 10 gigabases of sequence data.

A preliminary assembly yielded a mean coverage of below 15-fold, for
the \textit{A. crassus} derived contigs. This coverage is surprisingly
low given the large amount of input-data and I will need to construct
improved assemblies informed by the analysis of this preliminary
assembly. A seemingly trivial but nevertheless important prerequisite
for any high-throughput genomic sequencing project on a parasite was
the confirmation that genomic DNA could be obtained sufficiently clean
from other xenobiont DNA.

!!!Figure!!! GC-content and coverage for a preliminary genome
assembly. A preliminary assembly of roughly 10 Gb sequence data in
over 110 million reads. The analysis of GC-content and coverage
identifies host-contamination at higher GC, but lower
coverage. Coverage and GC-content separate two distinct data-sources:
a lower GC/higher coverage nematode subset and a higher GC/lower
coverage eel subset (confirmed by BLAST \cite{pmid2231712}). For this
sequencing library only 10-20\% of the reads are lost to eel-host
derived off-target data. The preliminary assembly was provided by
Sujai Kumar from Mark Blaxter's lab.

It has been possible to isolate roughly 1$\mu$g of genomic DNA from a
big individual worm. Only ca. 20\% of the DNA were derived from the
genome of the eel-host (see figure ). As only 300
ng of DNA material (with low amounts of contamination with host-blood)
are needed for RAD-sequencing, this can be achieved in most big
specimen of \textit{A. crassus}.

For both reference genome assembly and annotation and for the future
genome-scans I will continue to collaborate with Mark Blaxter's
laboratory at the University of Edinburgh. This group is actively
developing methods especially for RAD-sequencing and applying them to
questions in evolutionary model-species \cite{pmid21681211}.

Another useful strategy enabled by RAD-sequencing is the construction
of a physical genetic map in families of \textit{A. crassus}
(backcross is impossible). In addition to the population scale
approaches outlined above mapping of gene-expression quantitative
trait loci (eQTL) in mapping crosses between the two divergent
expression-phenotypes constitutes a promising route for the
investigation of genomic variants underlying the divergent
expression-phenotypes. Once transcripts can be anchored on genomic
contigs and linkage groups can be constructed to build a physical map
of the genome, a readout for hybrid F2 individuals could even be
transcriptomic data (RNA-seq) providing both genotype and
expression-phenotype.

A prime example for a research program on the evolution of
ecologically important traits is provided by the Stickleback
\textit{Gasterosteus aculeatus}: QTL-mapping has been performed to
fine-map the loss of lateral plates in freshwater populations
\cite{pmid18852878} and parallel adaptation has been investigated
using population genomics \cite{pmid20195501}. Both approaches used
RAD-sequencing. The sophistication and depth of insight available in
such an evolutionary model species is underlined by research on
adaptive reduction of pelvic structures, an evolutionary trajectory
shown to be favoured by the localisation of the underlying change in
an instable region of the genome \cite{pmid20007865}.

The hope to develop a similar research program based on the present
humble thesis seems presumptuous. Nevertheless, making full use of the
advances in sequencing technology it might be possible to rapidly gain
insight into the genomic organisation underlying contemporary
evolutionary change. The present RNA-seq data will be crucial in
achieving this goal, as it will be used to link expression phenotypes
with genomic sequence. An evolutionary leap in a core metabolic
process seems possible.

The ability to evolve via such a leap could even be an evolutionary
old trait retained in \textit{A. crassus} allowing it to colonise new
hosts. Therefore, comparative genomics relating population genetic
processes in \textit{A. crassus} to putatively adaptive change during
the acquisition of new host by other \textit{Anguillicola} species in
evolutionary time constitutes another route of research. If such a
link between microevolutionary processes in \textit{A. crassus} and
the evolution of \textit{Anguillicola}-species would exist, it would
provide general insight in the evolution of parasitic phenotypes.

\end{document}